\documentclass[style=ufz]{powerdot}

\usepackage[utf8]{inputenc}



\title{Your presentation title}
\author{John Doe}
\date{2013-04-30}

\begin{document}
\maketitle

\begin{slide}{overview}
\tableofcontents[content=sections]
\end{slide}

\section{motivation}
\begin{slide}{first slide title}
\begin{itemize}
 \item first item on first slide\pause
 \item second item
\end{itemize}
\end{slide}

\begin{slide}{second slide}
Your text may go here
\end{slide}

\section{second section}

\begin{slide}{how to compile}
\begin{itemize}
 \item 1st: compile the *.tex file with LaTeX, so you get a *.dvi file\pause\newline
 \item 2nd: compile the *.dvi with DVItoPS to get a postscript file\pause\newline
 \item 3nd: compile the *.ps file with PStoPDF to get a pdf (portable document format) file\pause\newline
 \item in KILE, you can do this at once with (a well configured) QuickBuild
\end{itemize}
\end{slide}
\begin{slide}{editing the template}
\begin{itemize}
 \item this powerdot presentation uses the ufz-style\pause\newline
 \item to alter the style, edit powerdot-ufz.sty
\end{itemize}
\end{slide}

\end{document}
